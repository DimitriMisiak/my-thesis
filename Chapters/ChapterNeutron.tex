% Chapter 1

\chapter{Neutron Measurements at IP2I cryogenic facility} % Main chapter title

\label{ChapterNeutron} % Change X to a consecutive number; for referencing this chapter elsewhere, use \ref{ChapterX}

%----------------------------------------------------------------------------------------
%	BEGING CHAPTER
%----------------------------------------------------------------------------------------

\section{Background}

As the Ricochet experiment is happening near a nuclear reactor, there is a risk that the bolometers will be blinded by the radioactive background of the site.
The radioactive background is mainly composed of particles inducing electronic recoil (such as gammas, electrons, muons, charged particles essentially) and neutron producing neutron recoil.
While the former background component can be discriminated thanks to technology used in the bolometers, the latter can not and thus constitute an unavoidable background for Ricochet which will limit the CENNS process measurement.
Therefore, it is vital to evaluate the neutron background on-site and understand its dependency in the energy.
In addition to estimating the limitations on the measurement, the study of the neutron background on-site will also be used in the design of the shielding.

The actual measurement of the neutron background at the ILL site was achieved with gaseous Helium-3 Tube Detector based on the "?neutron capture?". 
This detector technology is sensitive to thermal/fast? neutron at high energy >MeV which is order of magnitude superior to the energy range of the cryogenic germanium bolometers.
The estimation of the neutron background at ILL in the energy range of interest is then obtained by extrapolation of the neutron background at IP2I validated with relative measurements between the tube detector and the germanium detector technology.

\section{Experimental Setup}

The measure of the neutron background at the IP2I cryogenic facility was done with a newly designed low-threshold germanium detector of the RED series called RED80.

