% Chapter Conclusion

\chapter{Conclusion and Future Directions} % Main chapter title

\label{ChapterConclusion} % Change X to a consecutive number; for referencing this chapter elsewhere, use \ref{ChapterX}

%----------------------------------------------------------------------------------------
%	BEGING CHAPTER
%----------------------------------------------------------------------------------------

%What could be improved ?
%What has been achieved ?
%What is still unknown ?
%What should the next PhD Student focus on ?
%Will dark matter be unveiled ?
%Will CENNS be precisely-measured at ILL ?
%So much questions, so little inspiration...


\section*{Conclusion}

Le problème de la matière noire est l'un des plus grand défis de la physique moderne. Le résoudre passe par la recherche et la détection de particules WIMPs qui formerait un halo contenant notre galaxie. Leur important flux sur Terre permettrait de les détecter directement. 

L'expérience EDELWEISS étudie la diffusion élastique de WIMPs sur des noyaux de germanium. Pour cela, elle utilisent des détecteurs cryogéniques au sein du Laboratoire Souterrain de Modane. En vue de produire une nouvelle courbe d'exclusion dans les domaines des WIMPs de basse masse, une nouvelle campagne de R\&D a été lancée avec pour objectif actuel l'obtention de détecteur avec une résolution en voie chaleur de $100$eV.

Mon travail de stage au sein de l'équipe Manoir s'est axé sur le développement d'une nouvelle génération de détecteur cryogénique en se basant sur l'étude de deux détecteurs RED1 et RED10.  Dans un premier temps, j'ai construis un modèle électro-thermique de ces détecteurs permettant de prédire leur comportement dans l'état stationnaire, en régime temporel, et en régime fréquentiel. Ce dernier donne accès au calcul de la sensiblité du détecteur, et du bruit affectant sa mesure, permettant à terme l'évaluation de sa résolution en énergie. J'ai ensuite appliqué ce modèle, et analysé divers mesures expérimentales effectuées sur les deux détecteurs. Une analyse par Méthode de Monte Carlo avec Chaîne de Markov a permis de caractériser deux électronique, CUORE et EDELWEISS, et de caractériser thermiquement le nouveau détecteur RED10. La comparaison des données expérimentales et de simulation basées sur le modèle indique la présence de légères déviations mais affirment surtout une bonne adéquation entre théorie et expérience. Grâce à la performance du modèle, on a pu mettre en évidence la présence de phonons athermiques pour la première fois dans ce type de détecteur. Cela ouvre une voie d'optimisation encore jamais envisagée au sein de l'expérience EDELWEISS.

Un travail préliminaire d'optimisation du détecteur RED10 a été effectué. Il a été confirmé l'existence d'un courant de polarisation optimal suivant les conditions d'expérimentations, en particuliers la température du détecteur. Le travail d'optimisation futur se focalisera sur la géométrie du senseur thermique NTD et la configuration des liens thermiques.